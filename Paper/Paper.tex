\documentclass{article}

%% Packages 
\usepackage[english]{babel} % Language setting
\usepackage[a4paper,top=2cm,bottom=2cm,left=3cm,right=3cm,marginparwidth=1.75cm]{geometry} % Set page size and margins
\usepackage{amsmath}
\usepackage{graphicx}
\usepackage[colorlinks=true, allcolors=blue]{hyperref}
%%%%%%%%%%%%%%%%%%%%%%%%%%%%%%%%%%%%%%%%%%%%%%%%%%%%%%%%%%%%%%%%%%%%%%%%%%%%%%%%

%% Title
\title{Anonymisation of data for open science in psychology \\
\textcolor{red}{(Working title)} 
}
%% Authors
\author{Jiří Novák \and 
        Matthias Templ \and 
        Carolin Strobl
        }
%% Start of article %%%%%%%%%%%%%%%%%%%%%%%%%%%%%%%%%%%%%%%%%%%%%%%%%%%%%%%%%%%%
%%%%%%%%%%%%%%%%%%%%%%%%%%%%%%%%%%%%%%%%%%%%%%%%%%%%%%%%%%%%%%%%%%%%%%%%%%%%%%%%
\begin{document}
\maketitle

%% Abstract %%%%%%%%%%%%%%%%%%%%%%%%%%%%%%%%%%%%%%%%%%%%%%%%%%%%%%%%%%%%%%%%%%%
%%%%%%%%%%%%%%%%%%%%%%%%%%%%%%%%%%%%%%%%%%%%%%%%%%%%%%%%%%%%%%%%%%%%%%%%%%%%%%%
\begin{abstract}
\textcolor{red}{There is a great demand for making more research data openly available. The reproducibility of findings in psychology has been questioned, and more openly available data would make research more transparent and accessible. Unfortunately, many datasets cannot be publicly available for privacy reasons. On the other hand, researchers are increasingly more expected to share data with others for review, reanalysis, and reuse. To solve this issue, we suggest using methods of Statistical Disclosure Control for data anonymisation. These methods either modify or synthesise data so that it can be disclosed without revealing confidential information that may be associated with specific respondents. In this presentation, we review the work in this area and present different anonymisation approaches that can be used to protect data confidentiality. To prove the success of data anonymisation, data utility is discussed as the main objective to be maximised while providing data with a disclosure risk below certain limits. The concepts are illustrated by means of a practical application example.}
\end{abstract}

\keywords{\textbf{Keywords:}
open science, confidentiality, reproducibility, anonymization, synthetic data}

%% Introduction %%%%%%%%%%%%%%%%%%%%%%%%%%%%%%%%%%%%%%%%%%%%%%%%%%%%%%%%%%%%%%%
%%%%%%%%%%%%%%%%%%%%%%%%%%%%%%%%%%%%%%%%%%%%%%%%%%%%%%%%%%%%%%%%%%%%%%%%%%%%%%%
\section{Introduction}

\textcolor{red}{In this presentation, we review the work in this area and present different anonymisation approaches that can be used to protect data confidentiality.}

Open science is a movement that has been gaining strength and importance in recent years.
The movement's goal is to make available the results of scientific research, arising on the basis of public finances, so that they are reusable and replicable, traceable and transparent, trustworthy, and more financially effective, enabling a better connection of science across the world.
All started by Budapest Open Access Initiative~\cite{2012_OSI} in 2002, which was then supplemented with a set of rules in 2012~\cite{2012_OSI} and 2022~\cite{2022_OSI}.  This was followed by the Bethesda Statement on Open-Access Publishing~\cite{2003_Bethesda} in 2003 and Berlin Declaration on Open Access to Knowledge in the Sciences and Humanities~\cite{2003_Max_Planck}.

Budapest Open Access Initiative (BOAI)~\cite{2002_OSI} define \textit{Open Access} (OA) as \textit{"free availability on the public internet, permitting any users to read, download, copy, distribute, print, search, or link to the full texts of these articles, crawl them for indexing, pass them as data to software, or use them for any other lawful purpose, without financial, legal, or technical barriers other than those inseparable from gaining access to the internet itself"}.

In recommendations from 2012 BOAI~\cite{2012_OSI} further specified that \textit{"The worldwide campaign for OA to research articles should work more closely with the worldwide campaigns for OA to books, theses and dissertations, research data, government data, educational resources, and source code."}. In 2022, new recommendations~\cite{2022_OSI} for the next 10 years were released. Strong emphasis is put on Open infrastructure and its governance. It is recommended to host and publish OA texts, data, metadata, code, and other digital research outputs on open, community-controlled infrastructure. \textit{Open science} is made from open data, open metadata, open citations, open code, open protocols, open books, open theses and dissertations, open educational resources, open courseware, open digitization projects, open licenses, open standards, and open peer review.

From recent developments in recommendations to Open Science is necessary to mention Commission Recommendation (EU) 2018/790 of 25 April 2018 on access to and preservation of scientific information~\cite{2018_EU_2018/790}, the European University Association (EUA) Open Science Agenda 2025~\cite{2022_EUA} and UNESCO Recommendation on Open Science~\cite{2022_EUA}.

Recommendation 2018/790~\cite{2018_EU_2018/790} states that research data resulting from publicly funded research, including open access, should be findable, accessible, interoperable and re-usable, so-called \textit{FAIR principles}, unless this is unfeasible or conflicts with the future use of the research findings. There should be a strong emphasis on principle \textit{"As open as possible, as closed as necessary"}. 
EUA~\cite{2022_EUA} established its Open Science strategy for 2025 with three main priority areas: Open Access to scholarly outputs, FAIR research data, and institutional approaches to research assessment. In the vision of EUA 2025 are mentioned \textit{FAIR research data} as the norm in producing and sharing scientific knowledge and \textit{Open Science} as an integral part of research assessment practices.

UNESCO Recommendation~\cite{2022_EUA} defines \textit{Open Science} as \textit{"an
inclusive construct that combines various movements and practices aiming
to make multilingual scientific knowledge openly available, accessible and
reusable for everyone, to increase scientific collaborations and sharing of
information for the benefits of science and society, and to open the processes of scientific knowledge creation, evaluation and communication to societal actors beyond the traditional scientific community"}. In this recommendation, UNESCO promotes open access to scientific knowledge but equally emphasises the need for tools for pseudonymizing and anonymizing data so that as much data as possible can be shared as appropriate.




%% Data utility %%%%%%%%%%%%%%%%%%%%%%%%%%%%%%%%%%%%%%%%%%%%%%%%%%%%%%%%%%%%%%%
%%%%%%%%%%%%%%%%%%%%%%%%%%%%%%%%%%%%%%%%%%%%%%%%%%%%%%%%%%%%%%%%%%%%%%%%%%%%%%%
\section{Data utility}

\textcolor{red}{To prove the success of data anonymisation, data utility is discussed as the main objective to be maximised while providing data with a disclosure risk below certain limits.}

%% Acknowledgment %%%%%%%%%%%%%%%%%%%%%%%%%%%%%%%%%%%%%%%%%%%%%%%%%%%%%%%%%%%%%
%%%%%%%%%%%%%%%%%%%%%%%%%%%%%%%%%%%%%%%%%%%%%%%%%%%%%%%%%%%%%%%%%%%%%%%%%%%%%%%
\section*{Acknowledgment \& Disclosure} 
\subsection*{Acknowledgment} 
This work was funded by the Swiss National Science
Foundation with grant \textit{"Harnessing event and longitudinal data in industry and health sector through privacy preserving technologies"} (grant number 211751).

\subsection*{Disclosure of Interests} 
The authors have no competing interests to declare that are relevant to the content of this article. 

%% End of article %%%%%%%%%%%%%%%%%%%%%%%%%%%%%%%%%%%%%%%%%%%%%%%%%%%%%%%%%%%%%%
%%%%%%%%%%%%%%%%%%%%%%%%%%%%%%%%%%%%%%%%%%%%%%%%%%%%%%%%%%%%%%%%%%%%%%%%%%%%%%%%

%% References
\bibliographystyle{plain}
\bibliography{bib}
%%%%%%%%%%%%%%%%%%%%%%%%%%%%%%%%%%%%%%%%%%%%%%%%%%%%%%%%%%%%%%%%%%%%%%%%%%%%%%%%

\end{document}